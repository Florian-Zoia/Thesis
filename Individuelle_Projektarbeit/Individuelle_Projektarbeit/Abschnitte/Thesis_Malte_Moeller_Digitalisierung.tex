\documentclass[12pt]{article}
\usepackage{rotating}
\usepackage[utf8]{inputenc}
\usepackage[T1]{fontenc}
\usepackage[ngerman]{babel}
\usepackage{here}
\usepackage{listings} 
\lstset{language=XML} 
\lstset{numbers=left, numberstyle=\tiny, numbersep=5pt} \lstset{language=Perl}
\usepackage{xcolor}
\usepackage{tikz}
\usepackage{tabularx}
\usepackage{booktabs, multicol, multirow}
\usepackage{rotating}
\usepackage{bigstrut}
\usepackage{svg}
\usepackage{wrapfig}
\usepackage{pdfpages} 
\usepackage{floatflt}
\usepackage{amsmath, amssymb}
\usepackage[printonlyused]{acronym} 
\usepackage{textcomp}

\lstdefinelanguage{JavaScript}{
	keywords={typeof, new, true, false, catch, function, return, null, catch, switch, var, if, in, while, do, else, case, break},
	keywordstyle=\color{blue}\bfseries,
	ndkeywords={class, export, boolean, throw, implements, import, this},
	ndkeywordstyle=\color{darkgray}\bfseries,
	identifierstyle=\color{black},
	sensitive=false,
	comment=[l]{//},
	morecomment=[s]{/*}{*/},
	commentstyle=\color{purple}\ttfamily,
	stringstyle=\color{red}\ttfamily,
	morestring=[b]',
	morestring=[b]",
	extendedchars=true,
	basicstyle=\footnotesize\ttfamily,
	showstringspaces=false,
	showspaces=false,
	numbers=left,
	numberstyle=\footnotesize,
	numbersep=9pt,
	tabsize=2,
	breaklines=true,
	showtabs=false,
	captionpos=b
}

\lstset{literate=%
	{Ö}{{\"O}}1
	{Ä}{{\"A}}1
	{Ü}{{\"U}}1
	{ß}{{\"ss}}1
	{ü}{{\"u}}1
	{ä}{{\"a}}1
	{ö}{{\"o}}1
	{~}{{\textasciitilde}}1
}

%\usepackage{hyperref}
\usetikzlibrary{arrows,shapes,snakes,automata,backgrounds,petri}
\lstdefinestyle{base}{
  language=xml,
  emptylines=1,
  breaklines=true,
  basicstyle=\ttfamily\color{black},
  moredelim=**[is][\color{red}]{@}{@},
}


\newcommand{\autorPraxisbericht}{Malte Möller}
\newcommand{\titelPraxisbericht}{Entwicklung und Pilotierung eines digitalisierten Zeiterfassungsprozesses für das Unternehmen COM Software GmbH}
\newcommand{\titelVeranstaltung}{{\sl Exposé}}
\newcommand{\ersterBetreuer}{Prof.\ Dr.\ Martin Przewloka}
\newcommand{\zweiterBetreuer}{Prof.\ Dr.\ rer.\ nat.\ Martin Rupp}
\newcommand{\dritterBetreuer}{Helmut Röse}
\newcommand{\vierterBetreuer}{Dennis Panz}
\newcommand{\abgabeortPraxisbericht}{Frankfurt am Main}
\newcommand{\datumAbgabePraxisbericht}{1.\ April 2017}


\usepackage[ngerman]{babel}
\usepackage{times}
\usepackage{natbib}
\usepackage{pdfpages}
\usepackage{amssymb}
\usepackage{amsmath}
\usepackage{graphicx}
\usepackage{svg}
\usepackage{eurosym}
\usepackage{txfonts}
\usepackage{pifont}
\usepackage{url}
\usepackage{colortbl}
\urlstyle{tt}
\usepackage{tikz}
\usepackage{pgflibrarysnakes}
\usetikzlibrary{shadows,fadings}
\usetikzlibrary{decorations}
\usetikzlibrary{arrows} % LATEX and plain TEX when using Tik Z


%\usepackage[paper=a4paper, 
%%outer=15mm, 
%%inner=30mm, 
%%top=40mm, 
%%bottom=25mm, 
%bindingoffset=10mm]{geometry} 



\definecolor{white}{gray}{1.00}
\definecolor{black}{gray}{0.00}
\definecolor{skyblue}{cmyk}{0.4, 0.2, 0.0, 0.0}             % HKS44-40
\definecolor{blue}{cmyk}{1.0, 0.5, 0.0, 0.0}                % HKS44-100
\definecolor{lightblue}{cmyk}{0.7, 0.35, 0.0, 0.0}          % HKS44-70
\definecolor{darkblue}{rgb}{0.04, 0.16, 0.32}               % 
\definecolor{extradarkblue}{cmyk}{1.00, 0.70, 0.10, 0.50}   % HKS41-100
\definecolor{darkgreen}{cmyk}{1.0, 0.0, 0.9, 0.2}           % HKS57-100
\definecolor{green}{cmyk}{0.65, 0.0, 1.0, 0.0}              % HKS65-100
\definecolor{purple}{cmyk}{0.5, 1.0, 0.0, 0.0}              % HKS33-100
\definecolor{indigo}{cmyk}{0.8, 0.9, 0.0, 0.0}              % HKS36-100
\definecolor{gray}{gray}{0.59}
\definecolor{lightgray}{gray}{0.4}
\definecolor{darkgray}{gray}{0.50}
\definecolor{darkcyan}{cmyk}{0.87, 0.4, 0.4, 0.0}
\definecolor{cyan}{cmyk}{0.78, 0.19, 0.01, 0.0}
\definecolor{lightcyan}{cmyk}{0.39, 0.095, 0.005, 0.0}
\definecolor{extralightcyan}{cmyk}{0.16, 0.1, 0.0, 0.0}
\definecolor{beetleBlue}{RGB}{64,80,127}

\usepackage[
colorlinks=false,
urlcolor=black,
linkcolor=black
]{hyperref}

\setlength{\textwidth}{15.5cm}     %
\setlength{\textheight}{23cm}      %
\setlength{\evensidemargin}{0cm} %
\setlength{\oddsidemargin}{0.95cm}  %
\setlength{\topmargin}{-1cm}       %
\setlength{\topskip}{0cm}          %
\setlength{\headheight}{11pt}      %



%\setlength{\textwidth}{15.5cm}     %
%\setlength{\textheight}{23cm}      %
%\setlength{\evensidemargin}{1.5cm} %
%\setlength{\oddsidemargin}{1.5cm}  %
%\setlength{\topmargin}{-1cm}       %
%\setlength{\topskip}{0cm}          %
%\setlength{\headheight}{11pt}      %


\title{%
	\titelPraxisbericht\\%
	\vspace{8mm}{\large Bachelorthesis im Studiengang}\\%
	{\LARGE Bachelor Business Information Management}\\%
	{\large an der}\\%
	{\LARGE Provadis - School of International}\\%
	{\LARGE Management and Technology}\\%
}

\author{%
	{\normalsize vorgelegt von}\\%
	\vspace{4mm}\autorPraxisbericht\\%
	{\normalsize im Fach}\\
	{\LARGE \fach}\\
	\vspace{4mm}~\\{\normalsize Betreuer}\\%
	\ersterBetreuer}

\date{
	\vfill\abgabeortPraxisbericht, \datumAbgabePraxisbericht\\%
	~\\%
	\includegraphics[scale=.75]{ComLogo.png}\hfil
	\includegraphics[scale=.22]{ProvadisLogo.png}%
}


\newcommand{\Lab}[3] { %
 \put(#1,#2){\makebox(0,0){\shortstack[c]{#3}}}%
}

\newcommand{\lb}{\linebreak}%
\newcounter{wMinipage}%
\newcommand{\tiktxt}[4]{%
\setcounter{wMinipage}{#3*\real{1.3}}
\draw(#1 mm,#2 mm) node {\begin{minipage}{\thewMinipage mm}\begin{center}\setlength{\baselineskip}{2.5ex} #4\end{center}\end{minipage}};%
}%
\newcommand{\rtiktxt}[5]{%
\setcounter{wMinipage}{#3*\real{1.3}}
\draw(#1 mm,#2 mm) node[rotate=#5] {\begin{minipage}{\thewMinipage mm}\begin{center}\setlength{\baselineskip}{2.5ex} #4\end{center}\end{minipage}};%
}%





\newcommand{\spw}{StaffIT pro web}
\begin{document}

\maketitle
\thispagestyle{empty}

\newpage
\pagestyle{headings}



\pagenumbering{Roman}
\setcounter{page}{2}
\section*{Selbstständigkeitserklärung}
\addcontentsline{toc}{section}{Selbstständigkeitserklärung}
	Hiermit versichere ich, dass ich die vorliegende Thesis mit dem Titel\\
\begin{center}
	\glqq \titelPraxisbericht \grqq{}  \\
\end{center}
	selbstständig verfasst und keine anderen als die angegebenen Quellen und Hilfsmittel verwendet habe.\\

\begin{flushright}
	\abgabeortPraxisbericht, den \datumAbgabePraxisbericht \\
\vspace{1cm} 
\rule {6cm}{0.2mm}\\
	\autorPraxisbericht
\end{flushright}


\pagestyle{headings}

\newpage
\section*{}
	\tableofcontents
\newpage
\addcontentsline{toc}{section}{Abbildungsverzeichnis}
\listoffigures
\newpage
\addcontentsline{toc}{section}{Tabellenverzeichnis}
\listoftables
\addcontentsline{toc}{section}{Listing-Verzeichnis}
\lstlistoflistings

\newpage
\section*{Abkürzungsverzeichnis}
\addcontentsline{toc}{section}{Abkürzungsverzeichnis}
\begin{acronym}[etc]
\acro{API}{Application Pogramming Interface}
\acro{Batch}{Stapelverarbeitung}
\acro{CR}{Change Requests}
\acro{CRM}{Customer Relationship Management}
\acro{DBMS}{Datenbank Management System}
\acro{DWH}{Data Warehouse}
\acro{ETL}{Extract Transform Load}
\acro{FTP}{File Transfer Protocol}
\acro{GBAT}{Global Business Administration Tool}
\acro{GES}{Global Exhibitor Search}
\acro{GUI}{graphical user interface}
\acro{HTML}{Hypertext Markup Language}
\acro{HTTP}{Hypertext Transfer Protocol}
\acro{JSON}{JavaScript Object Notation}
\acro{R-DBMS}{relationales Datenbank Management System}
\acro{SFTP}{Secure File Transfer Protocol}
\acro{SLX2BAT}{Schnittstelle zwischen Saleslogix (InforCRM) und GBAT}
\acro{SP}{Sales Partner}
\acro{TG}{Tochtergesellschaft}
\acro{XML}{Extensible Markup Language}

\end{acronym}

\newpage
\pagenumbering{arabic}
\setcounter{page}{1}

\section{Einleitung}
Jede Organisation nutzt gewisse Prozesse um den betrieblichen Alltag und das dazugehörige Geschäft abzuwickeln. Diese Prozesse können in den unterschiedlichsten Formen abgebildet sein. Zum einen können die Prozesse einfach nur nach bestem Wissen und Gewissen gelebt werden, ganz nach dem Motto \flqq so wie es schon immer war\frqq. Eine erste Ausbaustufe ist der dokumentierte Prozess. Es ist also beschrieben, wie der jeweilige Ablauf normalerweise ausgestaltet sein soll. Im Optimalfall wurde dieser sogar schon analysiert und zu dem bestmöglichen Prozess optimiert. Doch reicht das für einen erfolgreichen Betrieb oder Bedarf es hier weiterer Aktionen?
\subsection{Erläuterung}
Es gibt immer Wege und Möglichkeiten etwas zu Verbessern. Mit dem Zeitalter der Digitalisierung sind unterstützende Software-Lösungen meist der Schlüssel zur Optimierung. Auf Prozesse bezogen kann dies beispielsweise die Softwareunterstützung einiger Teilaufgaben sein, oder sogar die gesamte Digitalisierung aller Prozessschritte in einem System. Um allerdings ein solches zu implementieren, sind wiederum Informationen von und für andere Systeme von großer Relevanz. Es müssen Schnittstellen erstellt und integriert werden. Um dies zu gewährleisten, muss der IST-Zustand mit den damit verbundenen Problemstellungen detailliert definiert sein. Diese Arbeit soll den IST-Prozess der Arbeitszeiterfassung im Gesamtkonzept eingliedern und bestmöglich beschreiben. Nach anschließender Analyse kann die Entwicklung eines Prototypen starten. 
\subsection{Betrieblicher Kontext} \label{betrieb}
Die \textit{COM Software GmbH} ist ein mittelständischer IT-Dienstleister mit Sitz im Rhein-Main-Gebiet. Der Branchenfokus der europaweiten Kunden liegt bei Banken und Finanzdienstleistern beziehungsweise deren teilweise ausgegliederten Rechenzentren und IT-Sparten. \\
Das Unternehmen wurde im Jahr 1996 als GmbH von zwei Gesellschaftern gegründet und wird von Herrn Helmut Röse geleitet. Es werden 2017 zum Jahresende 130 Mitarbeiter beschäftigt, davon ein Teil festangestellt und ein Teil freiberuflich. \\
Die Schwerpunkte der Beratungstätigkeit liegen bei der Softwareentwicklung und System\-administration sowie der Konzeption und Implementierung von individuellen Softwarearchitekturen. Abgerundet wird das Portfolio durch ein Angebot von Lösungen im Bereich der Beratungsleistung zur zielgerichteten Informationsversorgung von Organisationseinheiten im Rahmen von Data-Warehouse und Business-Intelligence. \\
Im Bereich der Digitalisierung ist die COM Software GmbH mit Sicherheit am Puls der Zeit, da viel Wert auf aktuelles Wissen gelegt wird. In den Bereichen Geschäftsprozessmanagement und Dokumentenmanagement sind bereits Systeme implementiert. Einige Prozesse, wie beispielsweise der Onboarding Prozess sind modelliert, in die Organisation integriert und werden gelebt. 
\subsection{Sensibilisierung des Themas}
Auch bei der COM Software GmbH gibt es ein hohes Maß an Optimierungspotential. Eine Prozesslandkarte ist erstellt, auf welcher die unterschiedlichsten betrieblichen Prozesse abgebildet sind. Einer dieser Prozesse ist die Arbeitszeiterfassung, welche in direktem Zusammenhang zu dem Rechnungslegungsprozess steht. Jeder Berater welcher im Kundenprojekt tätig ist, muss am Ende eines jeden Monats für seine Projekte Daten bei der COM Software GmbH für die Stundenabrechnung einreichen. Die Kerninformationen spiegeln sich in zwei Dokumenten, dem Stundennachweis und der Beraterrechnung wieder. Diese Dokumente werden zum aktuellen Zeitpunkt per Post oder Mail zur COM Software GmbH gesendet. Identifiziertes Problem ist, dass dieser Dateneingang kein einheitlichen Kommunikationsweg und kein einheitliches Format besitzt. Deshalb ist auch die einheitliche Weiterverarbeitung mit Systemunterstürzung ohne große manuellen Tätigkeiten unmöglich. Genau dieses Problem soll durch die Standardisierung und Digitalisierung des Informationseingangs mithilfe einer Applikation gelöst werden.
\subsection{Ziel der Arbeit}
Das Projekt hat ein klar definiertes Ziel. Es soll ein testfähiger Prototyp entstehen, welcher alle relevanten Daten für die Weiterverarbeitung in einem System konsolidiert. Jeder Berater soll die Informationen auf der einheitlichen Plattform eintragen. Eine hohe User Akzeptanz muss durch die Applikation sichergestellt werden. Deshalb sollen Grundinformationen von vornherein geladen und dem Nutzer zur Verfügung gestellt werden. Durch Schnittstellen zu Fremdsystemen sollen personalisierte Auswahlmöglichkeiten vorkonfiguriert und der gesamte Informationsgehalt aufbereitet werden. Durch Kennzahlen soll der IST-Prozess mit dem optimierten Prozess verglichen werden. 
\subsection{Aufbau der Arbeit}
Bei dem in Absatz \ref{betrieb} beschriebenen Onboarding Prozess ist schon ein spürbarer Mehrwert gegenüber der ursprünglichen Checkliste in Form eines Blatt Papiers zu erkennen. \\
Um den Prozess der Arbeitszeiterfassung und Verarbeitung zu Digitalisieren, wird der Inhalt dieser Arbeit in mehrere Stufen untergliedert. Zum Einen soll eine breite Grundlagenforschung und entsprechende Statistiken den aktuellen Stand der Digitalisierung, insbesondere in Bezug auf Geschäftsprozessmanagement und Dokumentenmanagement, darlegen. \\
Im nächsten Schritt soll der Ist-Prozess erhoben, bewertet und der optimierte Soll-Prozess entwickelt werden. Ist der Prozess final modelliert und mit dem Kunden abgestimmt, geht es in die Umsetzungsphase um einen Piloten zu entwickeln. In diesem konkreten Prozess gilt die Fachabteilung der Abrechnung als Kunde und wird in ständigem Austausch die Qualität der Lösung bestätigen. \\
Während der Umsetzung muss die Frage der richtigen Technologie anhand methodischer Analyse geklärt werden. Wichtig sind hier besonders die Aspekte Datensicherheit und Zugriffsschutz, da es sich um sensible Personenbezogene Daten handelt. \\
Abschließend wird der Pilot mit Kennzahlen, wie zum Beispiel den Prozesskosten und der Durchlaufzeit bewertet. Hierfür wird die Fachseite sowohl den Ist-Zustand als auch den Soll-Zustand nach erfolgreicher Implementierung der Lösung in Bezug auf die benötigten Ressourcen messen.
\subsection{Forschungsmethode}
Die Grundlage für ein erfolgreiches Projekt ist die korrekte und vor allem vollständige Aufnahme der IST-Situation. Nur anhand dieser kann ein hohes Optimierungsmaß erreicht werden. Sobald der IST-Zustand methodisch aufgenommen ist, wird dieser Analysiert und der SOLL-Zustand modelliert. Das Ergebnis soll anschließend in Form eines Prototypen entwickelt und vertestet werden.   



\input{./Abschnitte/Grundlagen.tex}

\input{./Abschnitte/Ist-Zustand.tex}


\input{./Abschnitte/Technologie.tex}

\input{./Abschnitte/prototyp.tex}

\newpage
\section{Fazit}
Dieses Kapitel reflektiert das durchgeführte Projekt. Die einzelnen Schritte werden in der Zusammenfassung chronologisch widergespiegelt. Die gewonnenen Ergebnisse für künftige Projekte und vor allem für das Management werden detailliert beschrieben. Der Ausblick bildet den Abschluss der Arbeit. Hier wird das geplante weitere Vorgehen beschrieben. Der Prototyp soll weiterentwickelt und anschließend für den produktiven Einsatz in das Unternehmen integriert werden.
\subsection{Zusammenfassung}
Das Management der COM Software GmbH hat in alle Bereiche des Unternehmens entsprechenden Einblick. Optimierungsmöglichkeiten sind vorhanden. Diese werden in internen Absprachen priorisiert. Anschließend fällt eine Einordnung und die jeweilige Projektentscheidung. So wurde die Thematik der Leistungsnachweise der im Projekt befindlichen Berater in diesen Plan aufgenommen und das Projekt aufgestellt.\\
Mit dem Projektstart wurden erste Termine mit der Fachabteilung abgestimmt, um die Thematik einzuordnen. Zu Beginn ist man von der reinen Arbeitszeiterfassung ausgegangen. Bereits in dem ersten Termin wurde deutlich, dass dies als Anforderung nicht ausreicht und die Thematik in einem größeren Kontext betrachtet werden muss. Somit startete die in Abschnitt \ref{IstProzess} beschriebene grobe Prozessanalyse des gesamten Rechnungslegungsprozesses. Anhand diesem wurde der Zusammenhang für alle Projektbeteiligten wesentlich deutlicher. Durch die anschließende Eingrenzung auf den Datenerhebungsprozess konnte das Ziel sehr deutlich eingegrenzt werden. Den Beteiligten ist nun klar, dass die Applikation nicht den gesamten Rechnungslegungsprozess abbilden kann. Hierfür gibt es bereits etablierte Applikationen. Diesen die nötigen Informationen aus einem zentralen System bereitzustellen ohne großen manuellen Aufwand ist somit Hauptmerkmal des Prototypen. Für die Reflektion der Projektarbeit wurden die Kennzahlen des Istprozesses aufgenommen.\\\\
\noindent
Anschließend beginnt die Entwicklungsphase des Prototypen. In dieser wurden die vorher definierten Anforderungen umgesetzt. In kontinuierlicher Rücksprache mit der Fachabteilung werden Teilergebnisse besprochen, sodass das Ziel des Prototypen nicht verfehlt wurde. Nach Tests und der Vorstellung der Applikation kam es in einem Termin zum Review des Projektergebnisses. Die positiven und negativen Anmerkungen wurden aufgenommen, Fragen bewertet und der Abschluss definiert. Hieraus entstanden sind die nächsten geplanten Schritte.

\subsection{Implikationen für Management \& Praxis}
Das Projekt hatte ein definiertes Projektteam bestehend aus Fachseite und IT-Entwicklung. Die mit Abstand wichtigste Projekterkenntnis ist, dass eine gute Kommunikation und die richtigen Personen in einem solchen Team der Schlüssel zum Erfolg sind. Es zählt somit nicht nur die Entwicklungskompetenz des repräsentativen Mitgliedes der IT, sondern auch dessen Fähigkeit, Zusammenhänge zu verstehen und in dem Kontext richtig und strukturiert aufzunehmen. Die Fachseite muss hinter dem Projekt stehen und dieses aktiv unterstützen. Ohne die gegebenen Einblicke in den aktiven Prozess und Alltag, die korrekte Messung der Ist-Kennzahlen und die Vorschläge für Verbesserungen wäre das Projekt nicht in diesem Stadium mit der gemessenen Qualität angekommen. Der zweite Punkt ist die ständige Qualitätssicherung durch die Fachabteilung, so dass keine Fehlentwicklung entstanden ist. Die Entscheidung des agilen Projektansatzes in Verbindung mit der gewählten Durchführungsmethodik Six Sigma ist rückblickend als gut zu bewerten. Allerdings muss dieser Ansatz auch weiter geführt werden und der sogenannte DMAIC-Zyklus als solcher periodisch wiederholt werden. Das Controlling der Prozesszahlen ist hier die wichtigste Kenngröße für das Management. 


\subsection{Ausblick}
Der entwickelte Prototyp ist Grundlage für die vollumfängliche Applikation. Es wird empfohlen, die Weiterentwicklung auch auf Projektebene zu strukturieren und einen entsprechenden Anforderungskatalog mit den gewonnenen Erkenntnissen und Funktionen zu erstellen. Der klare Nutzen für den Berater muss geschaffen werden. Dieser ist aktuell nur für die Mitarbeiter der COM Software GmbH vorhanden. Es wird klar davor gewarnt, dass die Berater den Prototypen in Anbetracht der aktuellen Funktionalitäten voraussichtlich nicht annehmen werden. Zu den fehlenden Benefits gehören die entsprechenden Übersichten und Auswertungsmöglichkeiten der Projekte, Abrechnungen, Zeitnachweise und Budgets der einzelnen Berater. Sind diese implementiert, kann man davon ausgehen, dass auch die Berater gerne mit dem Produkt arbeiten werden. Ein weiteres Kriterium ist das Dokumentenmanagement. Hier wurde begründet auf die Implementierung einer Integration zu dem revisionssicheren Dokumentenmanagementsystem ecoDMS verzichtet. Für die Eliminierung von weiteren manuellen Prozessschritten und der Zukunftsvision des papierlosen Büros muss für dieses identifizierte Problem eine Software gestützte Lösung entwickelt werden. \\
Sobald die dargestellten fehlenden Funktionalitäten implementiert sind, soll die auf zwei Monate beschränkte Beta-Phase der Applikation starten. Hierfür muss das Prozess-Portal im Internet verfügbar gemacht werden. Aktuell handelt es sich um eine Applikation, die lediglich im Intranet zur Verfügung steht. Das Portal soll sowohl von der Homepage als auch über den bereitgestellten Link der automatisch generierten Mails verfügbar sein. Als Anwendergruppe der Beta-Version werden die internen Mitarbeiter im Kundenprojekt vorgeschlagen. Die Rückmeldung soll anschließend bewertet werden und mögliche Änderungen eingearbeitet oder Bugs behoben werden. Ist dies geschehen soll die neuste Version für alle Berater zur Verfügung gestellt werden. Hierfür muss eine Dokumentation der Applikation erstellt werden und diese mit den generierten LogIn-Daten jedes Beraters versendet werden. \\
Der Six Sigma gestützte Lebenszyklus DMAIC soll fester Bestandteil der kontinuierlichen Weiterentwicklung sein. Besonders die letzte Phase \glqq Control\grqq{} wird häufig nicht komplett ausgeführt. Es gilt in weiteren Befragungen die Rückmeldung der Anwender einzuholen, Prozesskennzahlen zu messen und somit kontinuierlich in definierten Zeitintervallen weitere Optimierungsmaßnahmen durchzuführen und somit immer neue Versionen zu veröffentlichen. 





	\newpage

\pagenumbering{roman}
\setcounter{page}{4}
\bibliographystyle{hc-de} %

\addcontentsline{toc}{section}{Literatur} %
\bibliography{./Literaturverzeichnis} %


\newpage

\section{Anhang}
%
%\subsection{Sonderformen des Rechnungslegungsprozesses}
%\begin{figure}[H]
%	\centering
%	\includegraphics[width=15cm]{./Grafiken/PDF/rw_Schritt_3a_Rechnungsabwicklung_DZ_VR}
%	\caption{Rechnungsabwicklung DZ BANK AG \& VR Leasing AG}
%	\label{Sonderfall1}
%\end{figure} 
%
%
%\begin{figure}[H]
%	\centering
%	\includegraphics[width=15cm]{./Grafiken/PDF/rw_Schritt_3b_Rechnungsabwicklung_Hays}
%	\caption{Rechnungsabwicklung Hays AG}
%	\label{Sonderfall2}
%\end{figure} 



\end{document}