\newpage
\section{Fazit}
Dieses Kapitel reflektiert das durchgeführte Projekt. Die einzelnen Schritte werden in der Zusammenfassung chronologisch widergespiegelt. Die gewonnenen Ergebnisse für künftige Projekte und vor allem für das Management werden detailliert beschrieben. Der Ausblick bildet den Abschluss der Arbeit. Hier wird das geplante weitere Vorgehen beschrieben. Der Prototyp soll weiterentwickelt und anschließend für den produktiven Einsatz in das Unternehmen integriert werden.
\subsection{Zusammenfassung}
Das Management der COM Software GmbH hat in alle Bereiche des Unternehmens entsprechenden Einblick. Optimierungsmöglichkeiten sind vorhanden. Diese werden in internen Absprachen priorisiert. Anschließend fällt eine Einordnung und die jeweilige Projektentscheidung. So wurde die Thematik der Leistungsnachweise der im Projekt befindlichen Berater in diesen Plan aufgenommen und das Projekt aufgestellt.\\
Mit dem Projektstart wurden erste Termine mit der Fachabteilung abgestimmt, um die Thematik einzuordnen. Zu Beginn ist man von der reinen Arbeitszeiterfassung ausgegangen. Bereits in dem ersten Termin wurde deutlich, dass dies als Anforderung nicht ausreicht und die Thematik in einem größeren Kontext betrachtet werden muss. Somit startete die in Abschnitt \ref{IstProzess} beschriebene grobe Prozessanalyse des gesamten Rechnungslegungsprozesses. Anhand diesem wurde der Zusammenhang für alle Projektbeteiligten wesentlich deutlicher. Durch die anschließende Eingrenzung auf den Datenerhebungsprozess konnte das Ziel sehr deutlich eingegrenzt werden. Den Beteiligten ist nun klar, dass die Applikation nicht den gesamten Rechnungslegungsprozess abbilden kann. Hierfür gibt es bereits etablierte Applikationen. Diesen die nötigen Informationen aus einem zentralen System bereitzustellen ohne großen manuellen Aufwand ist somit Hauptmerkmal des Prototypen. Für die Reflektion der Projektarbeit wurden die Kennzahlen des Istprozesses aufgenommen.\\\\
\noindent
Anschließend beginnt die Entwicklungsphase des Prototypen. In dieser wurden die vorher definierten Anforderungen umgesetzt. In kontinuierlicher Rücksprache mit der Fachabteilung werden Teilergebnisse besprochen, sodass das Ziel des Prototypen nicht verfehlt wurde. Nach Tests und der Vorstellung der Applikation kam es in einem Termin zum Review des Projektergebnisses. Die positiven und negativen Anmerkungen wurden aufgenommen, Fragen bewertet und der Abschluss definiert. Hieraus entstanden sind die nächsten geplanten Schritte.

\subsection{Implikationen für Management \& Praxis}
Das Projekt hatte ein definiertes Projektteam bestehend aus Fachseite und IT-Entwicklung. Die mit Abstand wichtigste Projekterkenntnis ist, dass eine gute Kommunikation und die richtigen Personen in einem solchen Team der Schlüssel zum Erfolg sind. Es zählt somit nicht nur die Entwicklungskompetenz des repräsentativen Mitgliedes der IT, sondern auch dessen Fähigkeit, Zusammenhänge zu verstehen und in dem Kontext richtig und strukturiert aufzunehmen. Die Fachseite muss hinter dem Projekt stehen und dieses aktiv unterstützen. Ohne die gegebenen Einblicke in den aktiven Prozess und Alltag, die korrekte Messung der Ist-Kennzahlen und die Vorschläge für Verbesserungen wäre das Projekt nicht in diesem Stadium mit der gemessenen Qualität angekommen. Der zweite Punkt ist die ständige Qualitätssicherung durch die Fachabteilung, so dass keine Fehlentwicklung entstanden ist. Die Entscheidung des agilen Projektansatzes in Verbindung mit der gewählten Durchführungsmethodik Six Sigma ist rückblickend als gut zu bewerten. Allerdings muss dieser Ansatz auch weiter geführt werden und der sogenannte DMAIC-Zyklus als solcher periodisch wiederholt werden. Das Controlling der Prozesszahlen ist hier die wichtigste Kenngröße für das Management. 


\subsection{Ausblick}
Der entwickelte Prototyp ist Grundlage für die vollumfängliche Applikation. Es wird empfohlen, die Weiterentwicklung auch auf Projektebene zu strukturieren und einen entsprechenden Anforderungskatalog mit den gewonnenen Erkenntnissen und Funktionen zu erstellen. Der klare Nutzen für den Berater muss geschaffen werden. Dieser ist aktuell nur für die Mitarbeiter der COM Software GmbH vorhanden. Es wird klar davor gewarnt, dass die Berater den Prototypen in Anbetracht der aktuellen Funktionalitäten voraussichtlich nicht annehmen werden. Zu den fehlenden Benefits gehören die entsprechenden Übersichten und Auswertungsmöglichkeiten der Projekte, Abrechnungen, Zeitnachweise und Budgets der einzelnen Berater. Sind diese implementiert, kann man davon ausgehen, dass auch die Berater gerne mit dem Produkt arbeiten werden. Ein weiteres Kriterium ist das Dokumentenmanagement. Hier wurde begründet auf die Implementierung einer Integration zu dem revisionssicheren Dokumentenmanagementsystem ecoDMS verzichtet. Für die Eliminierung von weiteren manuellen Prozessschritten und der Zukunftsvision des papierlosen Büros muss für dieses identifizierte Problem eine Software gestützte Lösung entwickelt werden. \\
Sobald die dargestellten fehlenden Funktionalitäten implementiert sind, soll die auf zwei Monate beschränkte Beta-Phase der Applikation starten. Hierfür muss das Prozess-Portal im Internet verfügbar gemacht werden. Aktuell handelt es sich um eine Applikation, die lediglich im Intranet zur Verfügung steht. Das Portal soll sowohl von der Homepage als auch über den bereitgestellten Link der automatisch generierten Mails verfügbar sein. Als Anwendergruppe der Beta-Version werden die internen Mitarbeiter im Kundenprojekt vorgeschlagen. Die Rückmeldung soll anschließend bewertet werden und mögliche Änderungen eingearbeitet oder Bugs behoben werden. Ist dies geschehen soll die neuste Version für alle Berater zur Verfügung gestellt werden. Hierfür muss eine Dokumentation der Applikation erstellt werden und diese mit den generierten LogIn-Daten jedes Beraters versendet werden. \\
Der Six Sigma gestützte Lebenszyklus DMAIC soll fester Bestandteil der kontinuierlichen Weiterentwicklung sein. Besonders die letzte Phase \glqq Control\grqq{} wird häufig nicht komplett ausgeführt. Es gilt in weiteren Befragungen die Rückmeldung der Anwender einzuholen, Prozesskennzahlen zu messen und somit kontinuierlich in definierten Zeitintervallen weitere Optimierungsmaßnahmen durchzuführen und somit immer neue Versionen zu veröffentlichen. 
